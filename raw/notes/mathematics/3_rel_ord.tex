<!-- set parameters in the R code chunk below -->

```{r echo=FALSE}
unlockBinding("params", env = .GlobalEnv)
params$title <- "Lenguaje matemático. Conjuntos y números."
params$subtitle <- "Grado en Matemáticas UNED"
params$chapter <- "Relaciones de orden"
params$abstract <- ""
params$author <- "Gerlachito's Maths"
```

\fancypagestyle{fancy}{
  \fancyhead[OR,EL]{\large `r params$title`}
  \fancyhead[OL,ER]{\large `r params$chapter`}
  \fancyhead[C]{}
  \fancyfoot[OR,EL]{\thepage}
  \fancyfoot[OL,ER]{\large `r params$author`}
  \fancyfoot[C]{}
  \renewcommand{\headrulewidth}{1pt}
  \renewcommand{\footrulewidth}{1pt}
}

\fancypagestyle{firstpage}{
  \fancyhead[OR,EL]{\large `r params$title`}
  \fancyhead[OL,ER]{\large `r params$chapter`}
  \fancyhead[C]{}
  \fancyfoot[OR,EL]{\thepage}
  \fancyfoot[OL,ER]{\large `r params$author`}
  \fancyfoot[C]{}
  \renewcommand{\headrulewidth}{1pt}
  \renewcommand{\footrulewidth}{1pt}
}

\renewcommand{\maketitle}{
  \thispagestyle{firstpage}  
    \Huge
    \textbf{`r params$chapter`}
    \par
    \vspace{1cm}
    \large
    \textbf{`r params$abstract`}
    \par
    \vspace{1cm}
}

\pagestyle{fancy}
\newpage
\maketitle

\fontsize{0.4cm}{0.45cm}\selectfont

<!-- start writing your RMarkdown document below -->

Una relación $\mathcal{O}$ sobre un conjunto $\mathcal{U} \neq \emptyset$ se llama relación de orden si es reflexiva, antisimétrica y transitiva. El par $(\mathcal{U},\mathcal{O})$ se llama conjunto ordenado.

Podemos clasificar las relaciones de orden en dos subcategorías:

(1) Orden total: si $\mathcal{O} \cup \mathcal{R}^{-1} = \mathcal{U} \times \mathcal{U}$, i.e., $\forall x,y \in \mathcal{U}, x \mathcal{R} y \lor y \mathcal{R} x$.

(2) Orden parcial: si no es de orden total.

Sea $(\mathcal{U},\mathcal{O})$ o con otra notación ? un conjunto ordenado y $a.b \in \mathcal{U}$ con $a \mathcal{O} b$. Definimos:

(1) Intervalo abierto: $(a,b) := \{x \in \mathcal{U}: a \mathcal{O} x \mathcal{O} b\}$

(2) Intervalo cerrado: $[a,b] := \{x \in \mathcal{U}: a \mathcal{O} x \mathcal{O} b\}$

(3) Intervalos semiabiertos:

> $(a,b] := \{x \in \mathcal{U}: a \mathcal{O} x \mathcal{O} b\}$

> $[a,b) := \{x \in \mathcal{U}: a \mathcal{O} x \mathcal{O} b\}$

(4) Intervalo inicial abierto: $(\leftarrow, b] := \{x \in \mathcal{U}: x \mathcal{O} b\}$

(5) Intervalo final abierto: $[a, \rightarrow) := \{x \in \mathcal{U}: a \mathcal{O} x\}$

> **Def.: Cotas inferiores y superiores. Supremo e ínfimo. Máximos y mínimos.**

> Sea $\mathcal{U},\mathcal{O}$ un conjunto ordenado y $A \subset \mathcal{U}$. Tenemos:

> > (1) $M \in \mathcal{U}$ es una cota superior de $A$ si $a \mathcal{O} M, \forall a \in A$.

> > (2) $m \in \mathcal{U}$ es una cota inferior de A si $m \mathcal{O} a \forall a \in A$

> > (3) Llamamos supremo de A a la menor de las cotas superiores del conjunto A:

> > > $\alpha = sup(A) \in \mathcal{U} := min \{M \in \mathcal{U}: M$ cota superior de $A\}$

> > > Si $sup(A) \in A$, entonces lo llamamos máximo y escribimos $\alpha = max(A)$.

> > (4) Llamamos ínfimo de A a la mayor de las cotas inferiores de A:

> > > $\beta = inf(A) := max \{m \in \mathcal{U}: m$ cota inferior de $A\}$.

> > > Si $inf(A) \in A$, lo llamamos mínimo ($\beta = min(A)$).

Dado $A \subset \mathcal{U}$ con $(\mathcal{U}, \mathcal{O})$ conjunto ordenado decimos que A está:

(1) superiormente acotado si $\exists M \in \mathcal{U}$ cota superior de A.

(2) inferiormente acotado si $\exists m \in \mathcal{U}$ cota inferior de A.

(3) acotado si está superior e inferiormente acotado.

**Proposición:** Sea $\mathcal{U},\mathcal{O})$ un conjunto ordenado y $A \subset \mathcal{U}$. Entonces:

(1) Si existen máximo o mínimo de A, son únicos.

(2) Si existen supremo o ínfimo de A, son únicos.

> **Def.: Propiedad del buen orden**

> La propiedad del buen orden nos dice que si $(\mathcal{U},\mathcal{O})$ es un conjunto ordenado, entonces está bien ordenado si $\forall A \subset \mathcal{U}$ con $A \neq \emptyset$ se tiene que A posee mínimo (y dicho elemento lo llamamos primer elemento).

Por otro lado, decimos que $\mathcal{U},\mathcal{O})$ verifica la propiedad del supremo si $\forall A \subset \mathcal{U}$ acotado superiormente existe $sup(A)$.

Existen dos resultados que nos permiten caracterizar la existencia de supremo e ínfimo:

- Axioma del supremo:

<!--
dibujo de la caja inferior
-->

> $\alpha = sup(A) \iff \forall \epsilon > 0, \exists a \in A$ tal que $\alpha - \epsilon < a <= \alpha$.

- Axioma del ínfimo:

<!--
dibujo de caja superior
-->

> $\beta = inf(A) \iff \forall \epsilon > 0, \exists a \in A$ tal que $\beta <= a < \beta + \epsilon$

> **Def.: Maximal y minimal**

> Sea $(\mathcal{U},\mathcal{O})$ un conjunto ordenado y $A \subset \mathcal{U}$. Entonces:

> (1) Llamamos maximal de A a $M \in A$ tal que:

> > $\nexists a \in A, a \neq M$ con $M \mathcal{O} a$

> (2) Llamamos minimal de A a $m \in A$ tal que:

> > $\nexists a \in A, a \neq m$ con $a \mathcal{O} m$

**Nota:** En general puede haber varios elementos maximales o minimales. Si $\mathcal{O}$ es de orden total, entonces:

> máximo = maximal

> mínimo = minimal


