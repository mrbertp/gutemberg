<!-- set parameters in the R code chunk below -->

```{r echo=FALSE}
unlockBinding("params", env = .GlobalEnv)
params$title <- "Lenguaje matemático. Conjuntos y números."
params$subtitle <- "Grado en Matemáticas UNED"
params$chapter <- "Relaciones de equivalencia"
params$abstract <- ""
params$author <- "Gerlachito's Maths"
```

\fancypagestyle{fancy}{
  \fancyhead[OR,EL]{\large `r params$title`}
  \fancyhead[OL,ER]{\large `r params$chapter`}
  \fancyhead[C]{}
  \fancyfoot[OR,EL]{\thepage}
  \fancyfoot[OL,ER]{\large `r params$author`}
  \fancyfoot[C]{}
  \renewcommand{\headrulewidth}{1pt}
  \renewcommand{\footrulewidth}{1pt}
}

\fancypagestyle{firstpage}{
  \fancyhead[OR,EL]{\large `r params$title`}
  \fancyhead[OL,ER]{\large `r params$chapter`}
  \fancyhead[C]{}
  \fancyfoot[OR,EL]{\thepage}
  \fancyfoot[OL,ER]{\large `r params$author`}
  \fancyfoot[C]{}
  \renewcommand{\headrulewidth}{1pt}
  \renewcommand{\footrulewidth}{1pt}
}

\renewcommand{\maketitle}{
  \thispagestyle{firstpage}  
    \Huge
    \textbf{`r params$chapter`}
    \par
    \vspace{1cm}
    \large
    \textbf{`r params$abstract`}
    \par
    \vspace{1cm}
}

\pagestyle{fancy}
\newpage
\maketitle

\fontsize{0.45cm}{0.45cm}\selectfont

<!-- start writing your RMarkdown document below -->

Dado el conjunto universo $\mathcal{U} \neq \emptyset$, sea una relación $\mathcal{R} \subset \mathcal{U} \times \mathcal{U}$. Esta relación puede tener una o más de las siguientes propiedades:

(1) **Reflexiva**: si $\{(x,x): x \in \mathcal{U}\} \subset \mathcal{R}$, o en otros términos, si $x \mathcal{R} x, \forall x \in \mathcal{U}$.

(2) **Simétrica**: si $\mathcal{R}^{-1} \in \mathcal{R}$, o en otros términos, si $\forall x,y \in \mathcal{U}, x \mathcal{R} y \land y \mathcal{R} x$.

(3) **Antisimétrica**: si $\mathcal{R} \cap \mathcal{R} \subset \{(x,x): x \in \mathcal{U}\}$, o en otros términos si $\forall x,y \in \mathcal{U}, x \mathcal{R} y$ y $y \mathcal{R} x, x=y$.

(4) **Transitiva**: si $\mathcal{R} \circ \mathcal{R} \subset \mathcal{R}$, o en otros términos, $\forall x,y,z \in \mathcal{U}, x \mathcal{R} y, y \mathcal{R} z, x \mathcal{R} z$.

Sea $\mathcal{R}$ una relación definida en $\mathcal{U} \times \mathcal{U}$. Decimos que $\mathcal{R}$ es una relación de equivalencia si $\mathcal{R}$ es reflexiva, simétrica y transitiva.

> **Def.: Clase de equivalencia**

> > Sea $\mathcal{R}$ una relación de quivalencia y $x \in \mathcal{U}$.

> > Definimos una clase de equivalencia de $x$ como un conjunto tal que:

> > $[x] := \{y \in \mathcal{U}: x \mathcal{R} y\}$

> > (donde $x$ es un representante de la clase equivalente)

Supongamos que tenemos dos clases de equivalencia $[x], [\tilde{x}]$ de modo que:

$[x] \cap [\tilde{x}] \neq \emptyset \iff \exists y \in \mathcal{U}$ tal que $y \in [x], y \in [\tilde{x}]$

Esto implica que:

$x \mathcal{R} y$ y $\tilde{x} \mathcal{R} y$

$\tilde{x} \mathcal{R} y \iff y \mathcal{R} \tilde{x}$ ($\mathcal{R}$ es simétrica)

$x \mathcal{R} y, y \mathcal{R} \tilde{x} \implies x \mathcal{R} \tilde{x}$ ($\mathcal{R}$ es transitiva)

Por lo tanto:

$[x] = [\tilde{x}]$

**Nota**: Dos clases de equivalencia, o son disjuntas o son la misma.

> **Def.: Conjunto cociente**

> > Sea $\mathcal{R}$ una relación de equivalencia sobre $\mathcal{U} \neq \emptyset$.

> > Definimos el conjunto cociente de $\mathcal{U}$ por la relación $\mathcal{R}$ como:

> > $\mathcal{U}/\mathcal{R} := \{[x]: x \in \mathcal{U}\}$ 

Observemos que:

<!--
dibujo de territorios
-->

> **Def.: Partición**

> > Dado $\mathcal{U} \neq \emptyset$:

> > La partición ($\mathcal{P}$) de $\mathcal{U}$ es una familia de subconjuntos de $\mathcal{U}$ disjuntos dos a dos y cuya unión es el conjunto $\mathcal{U}$, es decir:

> > si $\mathcal{P} = \{{P_i}\}_{i \in I}$, entonces:

> > > i) $P_i \cap P_j = \emptyset, \forall i \neq j, i,j \in I$

> > > ii) $\bigcup_{i \in I} P_i = \mathcal{U}$

**Nota:** Observemos que si $\mathcal{R}$ es una relación de equivalencia sobre $\mathcal{U} \neq \emptyset$, entonces del conjunto cociente $\mathcal{U}/\mathcal{R}$ es una partición de $\mathcal{U}$.

> **Ejemplo:**

> > En $\mathbb{R}$ definimos la relación $x \mathcal{R} y \iff x-y \in \mathbb{Q}$.

> > Queremos ver que $\mathcal{R}$ es una relación de equivalencia y estudiar $\mathbb{R}/\mathcal{R}$.

> > (1) Reflexividad:

> > > $x \mathcal{R} x \iff x-x \in \mathbb{Q} \iff 0 \in \mathbb{Q}$. \checkmark 

> > (2) Simetría: partimos de $x \mathcal{R} y$, es decir:

> > > $x \mathcal{R} y \iff x-y \in \mathbb{Q} \implies -(x-y) \in \mathbb{Q} \iff y-x \in \mathbb{Q} \iff y \mathcal{R} x$. \checkmark

> > (3) Transitividad:

> > > $x \mathcal{R} y \iff x-y \in \mathbb{Q}$

> > > $y \mathcal{R} z \iff y-z \in \mathbb{Q}$

> > > Sabiendo que la suma de dos números racionales da un número racional:

> > > $a,b \in \mathbb{Q} \implies (a + b) \in \mathbb{Q}$, entonces:

> > > > $(x-y)+(y-z) \in \mathbb{Q} \iff x-z \in \mathbb{Q} \iff x \mathcal{R} z$ \checkmark
